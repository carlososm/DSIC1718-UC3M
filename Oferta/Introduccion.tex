% DOCUMENT

\section{Introducción}

\subsection{Objetivos de la propuesta}
\par En este documento de oferta se detalla el plan que se quiere seguir para desarrollar el software que CARSAFETY ha solicitado. Además, gracias a este documento, el cliente evaluará si la empresa S3 ha comprendido correctamente sus necesidades. Una vez finalizado el documento, CARSAFETY valorará si está interesado en la oferta que se le propone.
\par Tras una introducción en la que se expone la finalidad del trabajo y una breve descripción sobre la empresa desarrolladora del producto, se procederá a explicar los objetivos principales de la propuesta.
\par A continuación, se le presentará al cliente la metodología de trabajo con la que se va a llevar a cabo el proyecto y el equipo encargado de su desarrollo. También es importante definir un cronograma con las actividades a realizar y una buena planificación, para poder cumplir con los plazos que el cliente exija.
\par Finalmente, se detallarán los recursos necesarios para poder desarrollar un proyecto de calidad, así como un presupuesto para que el cliente pueda valorar si está interesado en la propuesta.

\subsection{Finalidad del trabajo a realizar}
\par Este proyecto consiste en desarrollar un software para proporcionar una solución de seguridad avanzada en los vehículos fabricados por el consorcio de fabricantes de vehículos CARSAFETY. Dicha solución tiene que contemplar varios subsistemas que proporcionarán una conducción más segura, actuando sobre el punto ciego, el cambio involuntario de carril, la superación de la velocidad máxima permitida, la pérdida de atención al volante, las llamadas de emergencia en caso de accidente y la prevención de colisiones.
\par En la conducción, la seguridad del propio conductor y la de los demás usuarios implicados tiene un papel fundamental, por lo que el software tiene que ser completamente fiable, ya que un error o retraso en el cómputo de los algoritmos implementados puede tener un coste muy elevado. Además, el sistema tiene que ser capaz de reaccionar ante situaciones excepcionales, por lo que también tendrá que ser robusto.

\subsection{Sobre S3}
\par La empresa S3: Smart Software Solutions fue creada en el año 2010 por 5 estudiantes del doble grado de ingeniería informática y administración y dirección de empresas de la Universidad Carlos III de Madrid. El objetivo principal de la empresa es proporcionar soluciones de calidad, eficientes e implementadas con la última tecnología al cliente, ajustándose lo máximo posible a sus necesidades.
\par Tras 7 años de experiencia, S3 ha conseguido diferenciarse de la competencia gracias a los buenos resultados obtenidos en proyectos anteriores, como por ejemplo SmartCity, un proyecto dedicado a proporcionar una infraestructura que garantice un desarrollo sostenible en el sistema de iluminación, el sistema de riego y  la gestión de residuos en Torrelodones. 
