\chapter{Propuesta de contrato}

En Colmenarejo, a 06 de Noviembre de 2017.
\begin{center}
  \textbf{REUNIDOS}
\end{center}

\par DE UNA  PARTE, \textbf{José Luis López Cuadrado}, mayor de edad, con D.N.I. número 01234567-X y en nombre y representación de Universidad Carlos III de Madrid, en adelante el \textbf{“CLIENTE”}, domiciliada en Getafe, calle Madrid nº126 C.P. 28903  y C.I.F. Q2818029G
\par DE OTRA  PARTE, J\textbf{orge Hevia Moreno}, mayor de edad, con D.N.I. número 100317565 y en nombre y representación de la mercantil RSPlusAgency, en adelante el \textbf{“PROVEEDOR”}, domiciliada en Colmenarejo, avenida Universidad Carlos III nº 22, C.P. 28270  y C.I.F. B2812345X.
El CLIENTE y el PROVEEDOR, en adelante podrán ser denominadas individualmente la “Parte” y conjuntamente las “Partes”, reconociéndose mutuamente capacidad jurídica y de obrar suficiente para la celebración del presente Contrato

\begin{center}
  \textbf{EXPONEN}
\end{center}

\begin{description} [style=multiline, leftmargin=3cm]
  \item[PRIMERO:] Que el CLIENTE está interesado en la contratación del servicio de:
  Diseño y desarrollo de un portal corporativo de gestión inmobiliaria de otros espacios para la venta y alquiler de los mismos.
  \item[SEGUNDO:] Que el PROVEEDOR es una empresa especializada en la prestación de servicios informáticos integrales.
  \item[TERCERO:] Que las Partes están interesadas en celebrar un contrato de diseño y desarrollo de programa informático en virtud del cual el PROVEEDOR preste al CLIENTE el servicio de diseño y desarrollo de un programa de software conforme a las necesidades específicas del negocio del CLIENTE.
  Que las Partes reunidas en la sede social del CLIENTE, acuerdan celebrar el presente contrato de DISEÑO Y DESARROLLO DE PROGRAMA INFORMÁTICO, en adelante el “Contrato”, de acuerdo con las siguientes
\end{description}

\begin{center}
  \textbf{CLÁUSULAS}
\end{center}

\renewcommand{\labelenumi}{%
 \textbf{\theenumi}.-
}

\renewcommand{\theenumii}{\arabic{enumii}}
\renewcommand{\labelenumii}{%
 \textbf{\theenumi}.\theenumii.-
}

\renewcommand{\theenumiii}{\arabic{enumiii}}
\renewcommand{\labelenumiii}{%
 \textbf{\theenumi}.\theenumii.\theenumiii.-
}

\begin{description}[style=nextline]
\item[PRIMERA.- OBJETO]
		En virtud de este Contrato el PROVEEDOR se obliga a prestar al CLIENTE el servicio de diseño y desarrollo de un programa de software conforme a las necesidades específicas del negocio del CLIENTE, en adelante el “Servicio”, en los términos y condiciones previstos en el Contrato y en todos sus Anexos.
\item[SEGUNDA.- TÉRMINOS Y CONDICIONES GENERALES Y ESPECÍFICOS DE PRESTACIÓN DE EL SERVICIO]
\newline
\begin{enumerate}
  \setcounter{enumi}{1}
  \item El servicio se prestará en los siguientes términos y condiciones generales:
  \begin{enumerate}
    \item El PROVEEDOR responderá de la calidad del trabajo desarrollado con la diligencia exigible a una empresa experta en la realización de los trabajos objeto del Contrato.
    \item El PROVEEDOR se obliga a gestionar y obtener, a su cargo, todas las licencias, permisos y autorizaciones administrativas que pudieren ser necesarias para la realización del servicio.
    \item El PROVEEDOR se hará cargo de la totalidad de los tributos, cualquiera que sea su naturaleza y carácter, que se devenguen como consecuencia del Contrato, así como cualesquiera operaciones físicas y jurídicas que conlleve, salvo el Impuesto sobre el Valor Añadido (IVA) o su equivalente, que el PROVEEDOR repercutirá al CLIENTE.
    \item El PROVEEDOR guardará confidencialidad sobre la información que le facilite el CLIENTE en o para la ejecución del Contrato o que por su propia naturaleza deba ser tratada como tal. Se excluye de la categoría de información confidencial toda aquella información que sea divulgada por el CLIENTE, aquella que haya de ser revelada de acuerdo con las leyes o con una resolución judicial o acto de autoridad competente. Este deber se mantendrá durante un plazo de tres años a contar desde la finalización del servicio.
    \item En el caso de que la prestación del servicio suponga la necesidad de acceder a datos de carácter personal, el PROVEEDOR, como encargado del tratamiento, queda obligado al cumplimiento de la Ley 15/1999, de 13 de diciembre, de Protección de Datos de Carácter Personal y del Real Decreto 1720/2007, de 21 de diciembre, por el que se aprueba el Reglamento de desarrollo de la Ley Orgánica 15/1999 y demás normativa aplicable.
    El PROVEEDOR responderá, por tanto, de las infracciones en que pudiera incurrir en el caso de que destine los datos personales a otra finalidad, los comunique a un tercero, o en general, los utilice de forma irregular, así como cuando no adopte las medidas correspondientes para el almacenamiento y custodia de los mismos. A tal efecto, se obliga a indemnizar al CLIENTE, por cualesquiera daños y perjuicios que sufra directamente, o por toda reclamación, acción o procedimiento, que traiga su causa de un incumplimiento o cumplimiento defectuoso por parte del PROVEEDOR de lo dispuesto tanto en el Contrato como lo dispuesto en la normativa reguladora de la protección de datos de carácter personal.
    A los efectos del artículo 12 de la Ley 15/1999, el PROVEEDOR únicamente tratará los datos de carácter personal a los que tenga acceso conforme a las instrucciones del CLIENTE y no los aplicará o utilizará con un fin distinto al objeto del Contrato, ni los comunicará, ni siquiera para su conservación, a otras personas. En el caso de que el PROVEEDOR destine los datos a otra finalidad, los comunique o los utilice incumpliendo las estipulaciones del Contrato, será considerado también responsable del tratamiento, respondiendo de las infracciones en que hubiera incurrido personalmente.
    El PROVEEDOR deberá adoptar las medidas de índole técnica y organizativas necesarias que garanticen la seguridad de los datos de carácter personal y eviten su alteración, pérdida, tratamiento o acceso no autorizado, habida cuenta del estado de la tecnología, la naturaleza de los datos almacenados y los riesgos a que están expuestos, ya provengan de la acción humana o del medio físico o natural. A estos efectos el PROVEEDOR deberá aplicar los niveles de seguridad que se establecen en el Real Decreto 1720/2007 de acuerdo a la naturaleza de los datos que trate.
    \item El PROVEEDOR responderá de la corrección y precisión de los documentos que aporte al CLIENTE en ejecución del Contrato y avisará sin dilación al CLIENTE cuando detecte un error para que pueda adoptar las medidas y acciones correctoras que estime oportunas.
    \item El PROVEEDOR responderá de los daños y perjuicios que se deriven para el CLIENTE y de las reclamaciones que pueda realizar un tercero, y que tengan su causa directa en errores del PROVEEDOR, o de su personal, en la ejecución del Contrato o que deriven de la falta de diligencia referida anteriormente.
    \item Las obligaciones establecidas para el PROVEEDOR por la presente cláusula serán también de obligado cumplimiento para sus posibles empleados, colaboradores, tanto externos como internos, y subcontratistas, por lo que el PROVEEDOR responderá frente al CLIENTE si dichas obligaciones son incumplidas por tales empleados.
  \end{enumerate}
  \item El PROVEEDOR prestará el servicio en los siguientes términos y condiciones específicos:
  \begin{enumerate}
    \item  El CLIENTE, que es quien mejor conoce sus necesidades, se obliga a prestar su colaboración activa al PROVEEDOR para el diseño y elaboración del programa contratado en todas sus fases, para llevar a buen término este contrato.
  	\item  Los empleados del CLIENTE y los técnicos del PROVEEDOR se deberán prestar colaboración en todo momento y hasta la finalización del presente contrato.
  	\item El CLIENTE presentará al PROVEEDOR un informe con las necesidades y previsiones que tenga a medio plazo, que sean necesarias para utilidad del programa de software.
  	\item Las características del programa, sus funciones y especificaciones técnicas se establecerán detalladamente en la sección \ref{sec:sw_req} de este documento. El contenido deberá presentarse en formato ejecutable Liferay.
  	\item El CLIENTE y el PROVEEDOR acordarán un plan de entregas donde se detallarán las fechas de entrega y el contenido de las diferentes versiones del programa.
  	\item El PROVEEDOR responderá de la autoría y originalidad del proyecto y del ejercicio pacífico de los derechos que cede al CLIENTE mediante el presente contrato.
  	\item El PROVEEDOR será el único responsable de la contratación de colaboraciones y de la relación con éstas, si las necesita para la elaboración y desarrollo de alguna parte del programa.
  	\item Si durante la realización del programa, cualquiera de las partes considerara introducir modificaciones en el programa, deberá notificarlo por escrito a la otra parte. El acuerdo de modificación se realizará por escrito, con todas las especificaciones técnicas y los nuevos plazos. El documento quedará unido al presente contrato.
  	\item Realizada la entrega del programa IRMASpace, el programa se instalará en el sistema informático del CLIENTE. Y el PROVEEDOR realizará las comprobaciones necesarias para verificar el buen funcionamiento del programa. Dichas pruebas deberán determinar la calidad, operatividad y desarrollo del conforme el presente contrato. El CLIENTE no podrá negarse u obstaculizar la realización de las comprobaciones.
  \item Entregada la versión definitiva, tras la comprobación, el CLIENTE tiene un plazo de 30 días para efectuar las reclamaciones y observaciones que considere para el buen funcionamiento del programa. El CLIENTE colaborará con el PROVEEDOR en el proceso de corrección o de reparación.
  \item Notificado por el CLIENTE un fallo al PROVEEDOR o el programa no supere el nivel mínimo de calidad exigido, el PROVEEDOR procederá a realizar las correcciones necesarias para llegar a la calidad exigida y el buen funcionamiento del programa, en el plazo señalado en la cláusula 6.3 de este contrato.
  \item Junto con la entrega definitiva el PROVEEDOR  entregará al CLIENTE los códigos y programas fuente de las aplicaciones desarrolladas para el CLIENTE, y toda la documentación técnica utilizada. El PROVEEDOR destruirá la información confidencial aportada por el cliente para facilitar o posibilitar la realización del proyecto.
  \item Transcurrido el plazo sin objeciones, el PROVEEDOR y el CLIENTE firmarán un documento de aceptación definitiva del programa. Dicho documento quedará unido al presente contrato.
  \item El PROVEEDOR garantiza el programa desarrollado por un período de 5 años. Durante ese tiempo el PROVEEDOR subsanará cualquier incidencia que se produzca en el programa desarrollado, conforme a la cláusula 6.3 de este contrato.
  \item El PROVEEDOR cede todos los derechos, sin reserva alguna, de propiedad sobre el programa IRMASpace al CLIENTE
  \item  El PROVEEDOR ejecutará el Contrato realizando de manera competente y profesional los Servicios, cumpliendo los niveles de calidad exigidos y realizará el proyecto completo.
  \end{enumerate}
\end{enumerate}

\item[TERCERA.- POLÍTICA DE USO]
 \par El CLIENTE  es el único responsable de determinar si el servicio que constituye el objeto de este Contrato se ajusta a sus necesidades.

\item[CUARTA.- PRECIO Y FACTURACIÓN]
\newline
\begin{enumerate}
  \setcounter{enumi}{3}
  \item El precio del Contrato es de 9.640,78 euros, IVA excluido.
  El pago de la factura se realizará, tras la aceptación por el CLIENTE del programa desarrollado, mediante transferencia bancaria a los 5 días de la fecha de recepción de la factura, en la siguiente cuenta corriente titularidad del PROVEEDOR: ES6621000418401234567891
\end{enumerate}
\item[QUINTA.- DURACIÓN DEL CONTRATO]
\par El plazo máximo de terminación del programa es de 2 meses a partir de la fecha referida en el encabezamiento del Contrato.
El retraso superior a 15 días será considerado como una incidencia crítica.

\item[SEXTA.- ACUERDO DE NIVEL DE SERVICIO]
\newline
\begin{enumerate}
  \setcounter{enumi}{5}
  \item
  \begin{enumerate}
    \item	El servicio prestado por el PROVEEDOR se realizará por personal especializado en cada materia. El personal del PROVEEDOR acudirá previsto de todo el material necesario, adecuado y actualizado, para prestar el servicio.
    \item	Una vez establecido el plan de entregas, no se admitirá una desviación superior al 10% respecto a los plazos fijados en dicho plan.
    \item	Las averías o el mal funcionamiento del servicio se comunicarán al PROVEEDOR en su domicilio a través de llamada telefónica o envío de fax.
    \item	Los problemas del puesto de trabajo se resolverán en un período máximo que dependerá de su gravedad.
    \begin{itemize}[-]
      \item	Se entiende por incidencia crítica: las incidencias que, en el marco de la prestación del servicio, afectan significativamente al CLIENTE, impidiendo al mismo el uso del programa.
      \item	Se entiende por incidencia grave: las incidencias que, en el marco de la prestación del servicio, afectan moderadamente al CLIENTE debido a una disponibilidad limita del producto.
      \item	Se entiende por incidencia leve: las incidencias que se limitan a entorpecer la prestación del servicio.
      La reparación se realizará en los siguientes períodos máximos  desde el aviso:
      \item	Incidencia crítica: 1 día.
      \item	Incidencia grave: 5 días.
      \item	Incidencia leve: 10 días
    \end{itemize}
  \end{enumerate}
\end{enumerate}

\item[SÉPTIMA.- MODIFICACIÓN]
\par Las Partes podrán modificar el contrato de mutuo acuerdo y por escrito.

\item[OCTAVA.- RESOLUCIÓN]
\par Las Partes podrán resolver el Contrato, con derecho a la indemnización de daños y perjuicios causados, en caso de incumplimiento de las obligaciones establecidas en el mismo.

\item[NOVENA.- NOTIFICACIONES]
	\par Las notificaciones que se realicen las Partes deberán realizarse por correo electrónico con acuse de recibo a las siguientes direcciones:
  \begin{itemize}[-]
    \item jillopez@inf.uc3m.es
    \item rsplus.software@gmail.com
  \end{itemize}

\item[DÉCIMA.- REGIMEN JURÍDICO]
\par El presente contrato tiene carácter mercantil, no existiendo en ningún caso vínculo laboral alguno entre el CLIENTE  y el personal del PROVEEDOR que preste concretamente el servicio.
\par Toda controversia derivada de este contrato o que guarde relación con él –incluida cualquier cuestión relativa a su existencia, validez o terminación- será resuelta mediante arbitraje DE DERECHO, administrado por la Asociación Europea de Arbitraje de Madrid (Aeade), de conformidad con su Reglamento de Arbitraje vigente a la fecha de presentación de la solicitud de arbitraje. El Tribunal Arbitral que se designe a tal efecto estará compuesto por un único árbitro experto y el idioma del arbitraje será el castellano. La sede del arbitraje será Madrid.

\end{description}
\par Y en prueba de cuanto antecede, las Partes suscriben el Contrato, en dos ejemplares y a un solo efecto, en el lugar y fecha señalados en el encabezamiento.
