\section{Resumen del Presupuesto}
\par A continuación, se muestra el presupuesto final del proyecto, siendo desarrollados los costes en la sección de \ref{sec:presupuesto} de una forma más detallada.
\subsection{Costes totales}
\par Los costes totales son para 21 semanas, aplicando un IVA del 21%
\begin{table}[H]
\begin{center}
\begin{tabular}{l c}
\textbf{DESCRIPCIÓN} & \textbf{TOTAL}\\ \hline \hline
Sueldo del equipo de trabajo & 2.251,70\\
Amortización de Equipos informáticos & 1.213,30\\
Software informático & 355,74\\
Material fungible & 285,90\\
Viajes y dietas & 400\\
Costes indirectos & 1.500\\ \hline \hline
\textbf{TOTAL} & \textbf{6.024,64}\\ \hline
\end{tabular}
\caption{Resúmen de costes totales.}
\label{tab:resumenTotal}
\end{center}
\end{table}

En esta tabla se muestra el coste del proyecto sin I.V.A, así como, el riesgo y el beneficio a obtener por la empresa.
\begin{table}[H]
\begin{center}
\begin{tabular}{l c}
\textbf{DESCRIPCIÓN} & \textbf{TOTAL}\\ \hline \hline
Coste del proyecto (sin IVA) &  6.024,64\\
Riesgo (en porcentaje) & 15\% \\
Beneficio (en porcentaje)** & 15\% \\ \hline \hline
\textbf{TOTAL (sin IVA)} & \textbf{7.967,59}\\ \hline \hline
IVA 21\% & 1.673,19 \\\hline \hline
\textbf{TOTAL} &  \textbf{9.640,78}\\ \hline
\end{tabular}
\caption{Riesgos y beneficios.}
\label{tab:total}
\end{center}
\end{table}
