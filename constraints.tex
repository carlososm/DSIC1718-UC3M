\chapter{Hipótesis y restricciones}\label{sec:constaints}

\par La primera hipótesis asumida en el desarrollo de \textit{IRMASpace} es que podrá ser integrada con los otros 5 subsistemas del proyecto. No obstante, cabe la posibilidad de que ello no suceda, de modo que el desarrollo de este subsistema quedaría completamente aislado del resto.

\par Por otro lado, las restricciones detectadas en el proyecto las podemos dividir en dos categorías. Aquellas dependientes del cumplimiento de la hipótesis enumerada en el párrafo anterior, y las restricciones generales del proyecto independientes del cumplimiento o no de la hipótesis. Comenzaremos con estas segundas.

\section{Restricciones generales}
\begin{itemize}
    \item El software deberá ser soportado por Liferay.
    \item Deberá ser accesible a través de distintos navegadores web (Google Chrome, Microsoft Edge, Firefox, Safari y Opera).
    \item Deberá ser accesible desde distinto dispositivos (móviles, ordenadores y tablets).
\end{itemize}

\section{Restricciones dependientes}
\par Si el proyecto y, en concreto, \textit{IRMASpace}, debieran acoparse al sistema junto con el resto del subsistema, encontraríamos las siguientes restricciones:
\begin{itemize}
    \item \textit{IRMASpace} deberá tener estructuras mínimas en común con el resto de subsistemas.
    \item Los nombres específicos del subsistema no deberán coincidir con los de otros subsistemas.
    \item Todos los subsistemas deberán ser soportados por Liferay.
\end{itemize}
