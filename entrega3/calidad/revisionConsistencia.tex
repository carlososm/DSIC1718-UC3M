\section{Revisión del Análisis de Consistencia}
\subsection{Revisión del Catálogo de Requisitos}
\par Adriana Lima como Responsable de Calidad confirmará que los requisitos se han especificado de forma estructurada, con un contenido preciso y completo tal y como se había establecido en el Plan de Aseguramiento de la Calidad. Nuestro responsable de Calidad se asegurará de que el catálogo ofrece las siguientes características:
\begin{itemize}[-]
  \item Identificación de absolutamente todos los requisitos de usuario.
  \item Coherencia entre el contenido del Catálogo y su objetivo.
  \item Cada requisito describe la funcionalidad que le corresponde.
  \item Correspondencia entre los requisitos del Catálogo y los requisitos obtenidos del usuario, por lo que el catálogo es completo.
  \item Descripción de los requisitos en un lenguaje claro, sin ambigüedades y, por tanto, preciso.
  \item El catálogo es auto descriptivo, ya que se describe su estructura y contenido.
  \item Se deberá realizar una matriz de trazabilidad para comprobar que todos los requisitos de usuario tienen asociado al menos un requisito de software, y de esta forma están presentes en el diseño del sistema.
\end{itemize}

\par Esta revisión se realizará salvo que se indique lo contrario por parte del Jefe de Proyecto, la semana del 7 de noviembre, cuando estén realizados los catálogos de Requisitos de software así como el de usuario.
