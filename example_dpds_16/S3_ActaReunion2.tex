\documentclass[10pt,a4paper,oldfontcommands]{plantillaDPDS}

\makepage{./img/s3-cabecera.png}
\pagestyle{myruled}

% DOCUMENT
\begin{document}
\pagecolor{fondo}
\color{principal}

% \title{PROYECTO: Dirección de Proyectos del Desarrollo del Software} %Con este nombre se guardará el proyecto en writeLaTex

\begin{titlingpage}
\begin{center}

%Figura
\begin{figure}[h]

\begin{center}
\includegraphics[width=1\textwidth]{./img/long_logo.png}
\end{center}
\end{figure}

\vfill

\textsc{\LARGE \textbf{EQUIPO DEL PROYECTO:}}\\[2em]
\textsc{\Large Carlos Olivares Sánchez Manjavacas}\\[1em]
\textsc{\Large Luis Cabrero García}\\[1em]
\textsc{\Large Jorge Hevia Moreno}\\[10em]

\textsc{\large \textbf{Grupo C2}}\\[1em]

\textsc{Doble grado en: Ingeniería Informática y ADE, Grado en Ingeniería Informática}\\[1em]

\textsc{Desarrollo de Sistemas de Información Coorporativos}\\[6em]


\end{center}

\textsc{Colmenarejo, España. \hspace*{\fill} Oct. 2017}


\end{titlingpage}


\chapter{Acta Reunión 2 - DPDS}

\begin{table}[h]
\begin{center}
\begin{tabular}{p{4cm} p{2cm} p{7,5cm}}

\multicolumn{3}{c}{\textbf{Dirección de Proyectos del Desarrollo del Software}} \\ \hline \hline
\textbf{Fecha} & \textbf{Hora} & \textbf{Lugar} \\
21/04/2017 & 17:00 & Aula 1.1.B10, UC3M - Campus Colmenarejo \\ \hline
\textbf{Reunión convocada por:} & \multicolumn{2}{p{9,5cm}}{Eduardo Herranz Sánchez} \\
\textbf{Tipo de reunión:} & \multicolumn{2}{p{9,5cm}}{Informativa} \\
\textbf{Apuntadores:} & \multicolumn{2}{p{9,5cm}}{Juan Abascal, Carlos Tormo, Alberto García y Daniel González} \\
\textbf{Redactor:} & \multicolumn{2}{p{9,5cm}}{Carlos Olivares} \\
\textbf{Asistentes:} & \multicolumn{2}{p{9,5cm}}{Grupo 50 - Doble Grado Ingeniería Informática y ADE.} \\ \hline

\end{tabular}
\end{center}
\end{table}



\begin{table}[h]
\begin{center}
\begin{tabular}{p{4cm} p{9,5cm}}

\multicolumn{2}{c}{\textbf{Revisión del Formato}} \\ \hline \hline
\textbf{Duración} & \textbf{Moderador} \\
5 minutos & Eduardo Herranz Sánchez \\ \hline
\multicolumn{2}{p{12,5cm}}{\tabitem Se debe mantener un formato uniforme no solo a lo largo de cada documento, sino entre todos los documentos entregados.} \\
\multicolumn{2}{p{12,5cm}}{\tabitem Se debe incluir en el índice todos los apartados que contenga la memoria, así como deben estar en la memoria todos los apartados del índice.} \\
\multicolumn{2}{p{12,5cm}}{\tabitem Es importante integrar las hojas de estado con el formato y el estilo del documento. El documento puede no incluir una hoja de estado, pero nunca está demás hacerlo.} \\ \hline

\end{tabular}
\end{center}
\end{table}



\begin{table}[h]
\begin{center}
\begin{tabular}{p{4cm} p{9,5cm}}

\multicolumn{2}{c}{\textbf{Revisión del Diagrama de Casos de Uso}} \\ \hline \hline
\textbf{Duración} & \textbf{Moderador} \\
5 minutos & Eduardo Herranz Sánchez \\ \hline
\multicolumn{2}{p{12,5cm}}{\tabitem Se debe tener un número adecuado de casos de uso. Más de quince resultan excesivos.} \\
\multicolumn{2}{p{12,5cm}}{\tabitem El número de actores debe ser superior a dos (menos está mal). En los trabajos realizados, hay entre uno y seis.} \\
\multicolumn{2}{p{12,5cm}}{\tabitem Se debe tener cuidado con el actor \textit{reloj}, pues es incorrecto incluirlo a no ser que se le otorguen funcionalidades externas.} \\
\multicolumn{2}{p{12,5cm}}{\tabitem El número de requisitos incluidos en los trabajos está entre 20 y 56. Lo correcto es un número intermedio, pues en el caso de 20 las descripciones son muy elevadas y se debe dividir y en el caso de 56 se trata de requisitos demasiado específicos.} \\ \hline

\end{tabular}
\end{center}
\end{table}




\begin{table}[h]
\begin{center}
\begin{tabular}{p{4cm} p{9,5cm}}

\multicolumn{2}{c}{\textbf{Revisión de los Casos de Uso de Alto Nivel}} \\ \hline \hline
\textbf{Duración} & \textbf{Moderador} \\
15 minutos & Eduardo Herranz Sánchez \\ \hline
\multicolumn{2}{p{12,5cm}}{\tabitem Precondiciones: es necesario integrarlas con el sistema. Ello implica indicar el estado del sistema antes del caso de uso.} \\
\multicolumn{2}{p{12,5cm}}{\tabitem Postcondiciones: es necesario indicar el estado del sistema tras el caso de uso. En ellas no se debe indicar las acciones de los actores, sino el estado final del sistema.} \\
\multicolumn{2}{p{12,5cm}}{\tabitem Los tiempos verbales usados pueden ser en presente. No merece la pena que sean en futuro.} \\
\multicolumn{2}{p{12,5cm}}{\tabitem Es necesario introducir, explicar y detallar los campos que se incluyen en cada una de los formatos de tablas usados.} \\
\multicolumn{2}{p{12,5cm}}{\tabitem Las descripciones de los casos de uso han de ser acordes con el número de requisitos funcionales incluidos en cada uno. Por ejemplo, si un caso de uso contiene nueve requisitos, la descipción debe ser larga para que sea útil.} \\
\multicolumn{2}{p{12,5cm}}{\tabitem Los casos de uso han de estar vinculados con los requisitos en la matriz de trazabilidad. Todos los requisitos han de estar contenidos en algún caso de uso. Los casos de uso que compartan requisitos tendrán una mayor dependencia entre ellos.} \\ \hline

\end{tabular}
\end{center}
\end{table}



\begin{table}[h]
\begin{center}
\begin{tabular}{p{4cm} p{9,5cm}}

\multicolumn{2}{c}{\textbf{Revisión de la Priorización}} \\ \hline \hline
\textbf{Duración} & \textbf{Moderador} \\
10 minutos & Eduardo Herranz Sánchez \\ \hline
\multicolumn{2}{p{12,5cm}}{\tabitem Se debe explicar y detallar la ponderación que se le da de cada factor a cada caso de uso.} \\
\multicolumn{2}{p{12,5cm}}{\tabitem Se pueden añadir o quitar factores de los aportados en la documentación de la asignatura, pero es necesario explicar el motivo.} \\
\multicolumn{2}{p{12,5cm}}{\tabitem La priorización no extrae las rodajas o iteraciones de manera perfecta, sino que se trata de una aproximación del 80\% aproximadamente. Para completar la división de iteraciones se deben tener en cuenta las dependencias. También se deben tener en cuenta las funcionalidades para dárselas al cliente paulatinamente. Además, no es recomendable introducir las dependencias como factor.} \\
\multicolumn{2}{p{12,5cm}}{\tabitem Se debe decidir el número de ciclos e indicar qué ciclos se incluirán en cada caso de uso. Todo ello debe ser explicado y detallado.} \\ \hline

\end{tabular}
\end{center}
\end{table}



\begin{table}[h]
\begin{center}
\begin{tabular}{p{4cm} p{9,5cm}}

\multicolumn{2}{c}{\textbf{Revisión del Plan de Gestión de la Calidad}} \\ \hline \hline
\textbf{Duración} & \textbf{Moderador} \\
10 minutos & Eduardo Herranz Sánchez \\ \hline
\multicolumn{2}{p{12,5cm}}{\tabitem Es imprescindible adaptar la plantilla inicial (basada en Métrica 3) a la metodología de Craig Larman. Así, se deben adaptar también los acrónimos y siglas.} \\
\multicolumn{2}{p{12,5cm}}{\tabitem Se incluyen cinco categorías diferentes de riesgos, aunque existe la posibilidad de añadir o eliminar categorías. No obstante, el número de riesgos detectados debe estar equilibrado con el número de categorías. Fundamentalmente, hay que centrarse en los riesgos tecnológicos, que deben ser entre el 30\% y el 40\%.} \\
\multicolumn{2}{p{12,5cm}}{\tabitem Es imprescindible adaptar las pautas de calidad a las referencias de Olmedilla.} \\
\multicolumn{2}{p{12,5cm}}{\tabitem No se debe copiar los índices de años anteriores, pues las plantillas se modifican. Se ha detectado copia pasando el documento por Turnitin.} \\ \hline

\end{tabular}
\end{center}
\end{table}



\begin{table}[h]
\begin{center}
\begin{tabular}{p{4cm} p{9,5cm}}

\multicolumn{2}{c}{\textbf{Revisión del Plan de Gestión de la Configuración}} \\ \hline \hline
\textbf{Duración} & \textbf{Moderador} \\
15 minutos & Eduardo Herranz Sánchez \\ \hline
\multicolumn{2}{p{12,5cm}}{\tabitem Las responsabilidades deben estar definidas. Se ha de indicar que actividades realiza cada persona del equipo.} \\
\multicolumn{2}{p{12,5cm}}{\tabitem En políticas, directivas y procedimientos aplicables es importante mencionar donde se almacena el proyecto, si se realiza en local o en algún servicio en la nube.} \\
\multicolumn{2}{p{12,5cm}}{\tabitem En políticas, directivas y procedimientos aplicables es importante mencionar si se utiliza algún software de control de versiones como \textit{git}.} \\
\multicolumn{2}{p{12,5cm}}{\tabitem En la jerarquía del producto, el gráfico incluido debe complementarse con una explicación. Si se decide no incluir los seis subsistemas en dicho esquema, hay que explicar la motivación de esta decisión.} \\
\multicolumn{2}{p{12,5cm}}{\tabitem En la identificación de los elementos de configuración se debe explicar el significado de cada acrónimo. Así mismo, se debe suponer que se hace codificación y pruebas, no únicamente pensar en la parte documental. Se deben incluir las librerías de código, scripts, etc.} \\
\multicolumn{2}{p{12,5cm}}{\tabitem No debemos olvidar todos los documentos que se deben entregar, junto con los diagramas de secuencia y el diagrama de transición de estados.} \\
\multicolumn{2}{p{12,5cm}}{\tabitem En el esquema de identificación se debe definir cómo se identifican cada uno de los elementos de configuración. Se debe incluir la línea base a la que pertenece, la fecha de creación y la fecha de modificación. Así mismo, se pueden poner esquemas dependiendo del elemento de configuración.} \\
\multicolumn{2}{p{12,5cm}}{\tabitem No sólo es necesario indicar los tipos de relaciones que existen, sino que se debe incluir también las relaciones generadas hasta el momento actual.} \\
\multicolumn{2}{p{12,5cm}}{\tabitem En cuanto a las líneas base, se deben indicar aquellas que estén abiertas o cerradas. Además, las fases de las entregas de clase son líneas base y se debe incluir una por iteración.} \\
\multicolumn{2}{p{12,5cm}}{\tabitem Se valora positivamente un procedimiento aplicable desarrollado a medida.} \\
\multicolumn{2}{p{12,5cm}}{\tabitem Los formularios deben ser on-line e incluir la dirección URL de los mismo. } \\ \hline

\end{tabular}
\end{center}
\end{table}



\begin{table}[h]
\begin{center}
\begin{tabular}{p{4cm} p{9,5cm}}

\multicolumn{2}{c}{\textbf{Revisión de la Estimación}} \\ \hline \hline
\textbf{Duración} & \textbf{Moderador} \\
10 minutos & Eduardo Herranz Sánchez \\ \hline
\multicolumn{2}{p{12,5cm}}{\tabitem No es admisible utilizar la presentación usada en las clases y modificarla ligeramente.} \\
\multicolumn{2}{p{12,5cm}}{\tabitem Los actores sin ajustar deben corresponderse con los indicados en el diagrama de casos de uso. Así mismo, se debe indicar cuáles son de complejidad media, complejos o simples.} \\
\multicolumn{2}{p{12,5cm}}{\tabitem Se deben indicar los casos de uso sin ajustar, así como tener en cuenta las iteraciones.} \\
\multicolumn{2}{p{12,5cm}}{\tabitem Es necesario explicar y comentar mejor los factores ambientales y técnicos. Esta explicación debe ser muy extensa.} \\
\multicolumn{2}{p{12,5cm}}{\tabitem El output del proceso de estimación es el tiempo de programación en horas hombre, se debe sacar el tiempo total del proyecto.} \\
\multicolumn{2}{p{12,5cm}}{\tabitem El output del proceso de estimación debe traducirse a meses y asignarle un tiempo a cada iteración. El tiempo dedicado a cada estimación debe estar justificado, por ejemplo basándose en la complejidad de los casos de uso.} \\
\multicolumn{2}{p{12,5cm}}{\tabitem Se debe definir lo que es una transacción.} \\ \hline

\end{tabular}
\end{center}
\end{table}



\begin{table}[h]
\begin{center}
\begin{tabular}{p{4cm} p{9,5cm}}

\multicolumn{2}{c}{\textbf{Revisión de la Planificación}} \\ \hline \hline
\textbf{Duración} & \textbf{Moderador} \\
10 minutos & Eduardo Herranz Sánchez \\ \hline
\multicolumn{2}{p{12,5cm}}{\tabitem Todo lo explicado, indicado y utilizado en la planificación debe estar integrado en el documetno PDF.} \\
\multicolumn{2}{p{12,5cm}}{\tabitem Los resultados obtenidos en la estimación y la planificación no tienen porque coincidir. La planificación incluye la estimación y el Estudio de Viabilidad del Sistema entre otros.} \\
\multicolumn{2}{p{12,5cm}}{\tabitem En la planificación Gannt es necesario incluir el porcentaje previsto de realización de cada una de las personas involucradas.} \\
\multicolumn{2}{p{12,5cm}}{\tabitem El PERT entregado debe ser visible (tamaño adecuado).} \\ \hline

\end{tabular}
\end{center}
\end{table}



\begin{table}[h]
\begin{center}
\begin{tabular}{p{4cm} p{9,5cm}}

\multicolumn{2}{c}{\textbf{Revisión del Informe Quinquenal de Seguimiento}} \\ \hline \hline
\textbf{Duración} & \textbf{Moderador} \\
10 minutos & Eduardo Herranz Sánchez \\ \hline
\multicolumn{2}{p{12,5cm}}{\tabitem Es imprescindible realizar el Plus-Minus-Interesting (PMI).} \\
\multicolumn{2}{p{12,5cm}}{\tabitem Es necesario indicar y plantear los problemas que puedan salir.} \\
\multicolumn{2}{p{12,5cm}}{\tabitem La adaptación del IQS se debe adaptar a la metodología de Craig Larman y no a Métrica 3.} \\ \hline

\end{tabular}
\end{center}
\end{table}



\end{document}
