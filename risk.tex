\section{Análisis de riesgos}\label{sec:riesgos}
\par Para que el proyecto se realice en el tiempo especificado y entregando la máxima calidad posible, es necesario identificar los riesgos que pueden ir lastrando o puedan provocar la cancelación de éste, para así prevenirlos.
\par En este documento se recogerá una lista de riesgos comunes a este tipo de proyectos, que se irá ampliando en futuras revisiones (ver ~\ref{sec:calidad} del Plan de Gestión de la Calidad) según se descubran nuevos riesgos. Para que sea más legible se han categorizado en:

\begin{itemize}[-]
\item \textbf{Tamaño del producto:} riesgos que tienen que ver con el tamaño del software.
\item \textbf{Impacto en el negocio:} riesgos asociados con  la gestión o el mercado.
\item \textbf{Características del cliente:} riesgos asociados con el cliente y la comunicación entre éste y el desarrollador.
\item \textbf{Tecnología a construir:} riesgos debidos a la complejidad del sistema a construir y los componentes que lo sostienen.
\item \textbf{Tamaño y experiencia de la plantilla:} riesgos debidos a la experiencia técnica.
\end{itemize}

\par Para comprender mejor el riesgo, se han declarado en unas tablas con las siguientes características:
\begin{itemize}
	\item \textbf{Coste:} el coste en el proyecto, tanto monetario como en horas de esfuerzo.
	\item \textbf{Tiempo:} la cantidad de tiempo que podría retrasar el avance del proyecto.
	\item \textbf{Alcance:} la cantidad de impacto en las áreas del proyecto. Un riesgo que tiene mucho alcance afectará a más áreas que uno que tiene un menor alcance.
	\item \textbf{Calidad:} cuánto afecta a la calidad final del producto.
\end{itemize}

\subsection{Tamaño del producto}
\begin{itemize}
	\item \textbf{Riesgo-01: subestimar el tamaño del producto}
	\begin{table}[h]
	\begin{center}
	\begin{tabular}{ l l l l l l }
	\hline
	& Muy Bajo & Bajo & Moderado & Alto & Muy Alto \\ \hline \hline
	Coste &  &  &  & X &  \\ \hline
	Tiempo &  &  &  & X &  \\ \hline
	Alcance &  &  &  &  & X \\ \hline
	Calidad &  &  &  &  & X \\ \hline
	\end{tabular}
	\caption{Riesgo-01.}
	\label{Riesgo-01}
	\end{center}
	\end{table}
\end{itemize}

\subsection{Impacto en el negocio}
\begin{itemize}
	\item \textbf{Riesgo-02: La documentación del proyecto realizada por el cliente no tiene la suficiente calidad}
	\begin{table}[H]
	\begin{center}
	\begin{tabular}{ l l l l l l }
	\hline
	& Muy Bajo & Bajo & Moderado & Alto & Muy Alto \\ \hline \hline
	Coste &  &  & X &  &  \\ \hline
	Tiempo &  &  & X &  &  \\ \hline
	Alcance & X &  &  &  &  \\ \hline
	Calidad &  &  &  & X &  \\ \hline
	\end{tabular}
	\caption{Riesgo-02.}
	\label{Riesgo-02}
	\end{center}
	\end{table}
	\item \textbf{Riesgo-03: Costes asociados a retrasos}
	\begin{table}[H]
	\begin{center}
	\begin{tabular}{ l l l l l l }
	\hline
	& Muy Bajo & Bajo & Moderado & Alto & Muy Alto \\ \hline \hline
	Coste &  &  & X &  &  \\ \hline
	Tiempo &  &  &  & X &  \\ \hline
	Alcance &  &  &  & X &  \\ \hline
	Calidad &  &  & X &  &  \\ \hline
	\end{tabular}
	\caption{Riesgo-03.}
	\label{Riesgo-03}
	\end{center}
	\end{table}
\end{itemize}

\subsection{Características del cliente}
\begin{itemize}
	\item \textbf{Riesgo-04: El cliente no tiene una idea clara del producto}
	\begin{table}[H]
	\begin{center}
	\begin{tabular}{ l l l l l l }
	\hline
	& Muy Bajo & Bajo & Moderado & Alto & Muy Alto \\ \hline \hline
	Coste &  &  &  & X &  \\ \hline
	Tiempo &  &  &  & X &  \\ \hline
	Alcance &  & X &  &  &  \\ \hline
	Calidad &  &  & X &  &  \\ \hline
	\end{tabular}
	\caption{Riesgo-04.}
	\label{Riesgo-04}
	\end{center}
	\end{table}
	\item \textbf{Riesgo-05: El cliente no comprende el proceso de creación del software}
	\begin{table}[H]
	\begin{center}
	\begin{tabular}{ l l l l l l }
	\hline
	& Muy Bajo & Bajo & Moderado & Alto & Muy Alto \\ \hline \hline
	Coste &  &  & X &  &  \\ \hline
	Tiempo &  &  &  & X &  \\ \hline
	Alcance &  & X &  &  &  \\ \hline
	Calidad &  &  & X &  &  \\ \hline
	\end{tabular}
	\caption{Riesgo-05.}
	\label{Riesgo-05}
	\end{center}
	\end{table}
	\item \textbf{Riesgo-06: El cliente no tiene una vía clara y directa para comunicarse con el fabricante}
	\begin{table}[H]
	\begin{center}
	\begin{tabular}{ l l l l l l }
	\hline
	& Muy Bajo & Bajo & Moderado & Alto & Muy Alto \\ \hline \hline
	Coste &  &  & X &  &  \\ \hline
	Tiempo &  &  &  & & X  \\ \hline
	Alcance &  & & & X  &  \\ \hline
	Calidad &  &  & & & X  \\ \hline
	\end{tabular}
	\caption{Riesgo-06.}
	\label{Riesgo-06}
	\end{center}
	\end{table}
\end{itemize}

\subsection{Tecnología a construir}
\begin{itemize}

	\item \textbf{Riesgo-06: Falta de documentación acerca de la tecnología}
	\begin{table}[H]
	\begin{center}
	\begin{tabular}{ l l l l l l }
	\hline
	& Muy Bajo & Bajo & Moderado & Alto & Muy Alto \\ \hline \hline
	Coste &  &  & X & &  \\ \hline
	Tiempo &  &  &  &  & X \\ \hline
	Alcance &  &  & X & &  \\ \hline
	Calidad &  &  &  & X &  \\ \hline
	\end{tabular}
	\caption{Riesgo-06.}
	\label{Riesgo-06}
	\end{center}
	\end{table}
	\item \textbf{Riesgo-07: Limitaciones de la tecnología a utilizar}
	\begin{table}[H]
	\begin{center}
	\begin{tabular}{ l l l l l l }
	\hline
	& Muy Bajo & Bajo & Moderado & Alto & Muy Alto \\ \hline \hline
	Coste &  &  &  & X &  \\ \hline
	Tiempo &  &  &  &  & X \\ \hline
	Alcance &  &  &  & X &  \\ \hline
	Calidad &  &  &  & X &  \\ \hline
	\end{tabular}
	\caption{Riesgo-07.}
	\label{Riesgo-07}
	\end{center}
	\end{table}
\end{itemize}

\subsection{Tamaño y experiencia de la plantilla}
\begin{itemize}
	\item \textbf{Riesgo-09: Falta de experiencia con la tecnología}
	\begin{table}[H]
	\begin{center}
	\begin{tabular}{ l l l l l l }
	\hline
	& Muy Bajo & Bajo & Moderado & Alto & Muy Alto \\ \hline \hline
	Coste &  &  &  & X &  \\ \hline
	Tiempo &  &  &  &  & X \\ \hline
	Alcance &  &  &  & X &  \\ \hline
	Calidad &  &  &  & X &  \\ \hline
	\end{tabular}
	\caption{Riesgo-09.}
	\label{Riesgo-09}
	\end{center}
	\end{table}
\end{itemize}
