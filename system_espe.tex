\chapter{Especificaciones del sistema}
\par En esta sección se incluyen los requisitos, tanto funcionales como no funcionales, de usuario elicitados por RSPlusAgency a partir de la documentación original.

\section{Requisitos Funcionales de Usuario}
\par Los requisitos de usuario funcionales son:
\begin{itemize}
	\item \textbf{Poder crear o eliminar un usuario} a partir de un nombre, un alias y una dirección de correo electrónico.
	\item \textbf{Poder alquilar, comprar o vender un activo} que ofrezca el portal web.
	\item \textbf{Poner en alquiler o venta un activo} a través del portal web y un comercial.
	\item \textbf{Poder coordinar los distintos equipos de ventas} dentro de la empresa a traves del portal web como si de una intranet se tratase.
\end{itemize}

\section{Requisitos No Funcionales de Usuario}
\par Los requisitos de usuario no funcionales son:
\begin{itemize}
	\item \textbf{Una interfaz con una alta usabilidad} que sea fácil de manejar y aprender.
	\item \textbf{El portal deberá ser responsive} y ser capaz de adaptarse a todos los tamaños de pantalla, de forma que alcance a un mayor público.
	\item \textbf{La carga de imágenes deberá ser veloz} y dependiendo del tamaño de pantalla consumir menos ancho de banda para favorecer a los usuarios móviles.
	\item \textbf{El portal deberá soportar varios idiomas} para que el cliente pueda alcanzar a la máxima población de usuarios posible.
\end{itemize}
